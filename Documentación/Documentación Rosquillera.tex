\documentclass[10pt,a4paper,titlepage]{report}
\usepackage[utf8]{inputenc}
\usepackage{amsmath}
\usepackage{amsfonts}
\usepackage{amssymb}

\usepackage{titling}
\renewcommand{\contentsname}{Indice de contenidos}
\newcommand{\subtitle}[1]{
  \posttitle{
    \par\end{center}
    \begin{center}\large#1\end{center}
    \vskip0.5em}
}
\newcommand{\mychapter}[2]{
    \setcounter{chapter}{#1}
    \setcounter{section}{0}
    \chapter*{#2}
    \addcontentsline{toc}{chapter}{#2}
}
\newcommand{\mysection}[2]{
    \setcounter{section}{#1}
    \setcounter{subsection}{0}
    \section*{#2}
    \addcontentsline{toc}{section}{#2}
}

\author{Autor: Yawin}
\date{}
\title{La Rosquillera Framework}
\subtitle{Documentación v1.3$\beta$}

\begin{document}
\maketitle
\tableofcontents

\mychapter{0}{Introducción}
A lo largo de los casi diez años que llevo programando videojuegos y aplicaciones gráficas me he encontrado con que los motores y frameworks libres no reunen las características que necesito y los que sí las reúnen no son libres. El mundo del software libre tiene sus grandes logros (como los sistemas GNU, navegadores como Firefox y Midori, editores de vídeo como Natron, y un largo etcétera) pero cuando nos enfocamos en videojuegos nos encontramos con problemas: o no tiene soporte 3D, o su editor da problemas constantes, o su curva de dificultad es inhumana para un autodidacta.
\\
\\Por todo ello, y con la finalidad añadida de aprender, decidí desarrollar un framework para hacer videojuegos y aplicaciones gráficas que no tuviera una excesiva curva de dificultad, que tuviera la versatilidad que requiere la programación de videojuegos y, sobre todo, que respetara la libertad de sus usuarios.

\mysection{0}{El paradigma}
A lo largo de mi aprendizaje en el desarrollo de videojuegos, he conocido varias herramientas que me han influído en mi forma de entender los motores:

\begin{itemize}
\item Creo que todos los elementos del motor deben estar sincronizados a través de un gestor que los actualice uno a uno. Para ello entiendo la necesidad de una clase base "proceso" que agrupe a todos los elementos del juego.
\item Todas las clases del motor deben ser virtuales para que el usuario pueda crear sus propias versiones sin tener que modificar el framework base.
\item La gestión de recursos, así como el trabajo con las clases del framework, debe ser sencilla.
\end{itemize}

\mysection{1}{Dependencias}
La Rosquillera Framework requiere las siguientes librerías:
\begin{itemize}
\item SDL2
\item SDL2 Image
\item SDL2 Mixer
\item SDL2 TTF
\end{itemize}

\mychapter{1}{La gestión de procesos}
La Rosquillera entiende cada elemento del juego como un proceso; un objeto actualizable con toda una batería de funciones y miembros comunes

\mysection{0}{El task manager}
Blah, blah, blah, blah, blah

\mysection{1}{Los procesos}
Blah, blah, blah, blah, blah

\mychapter{2}{El motor principal}
bla bla bla

\mychapter{3}{Subsistemas}
bla bla bla

\mysection{0}{La ventana}
Blah, blah, blah, blah, blah

\mysection{1}{El reloj}
Blah, blah, blah, blah, blah

\mysection{2}{Matemáticas}
Blah, blah, blah, blah, blah

\mysection{3}{Parallax}
Blah, blah, blah, blah, blah

\mysection{4}{Sonido}
Blah, blah, blah, blah, blah

\mysection{5}{Subsistema de 3D}
Blah, blah, blah, blah, blah

\mychapter{4}{ToDo}
bla bla bla

\mychapter{5}{Consideraciones finales}
bla bla bla

\end{document}